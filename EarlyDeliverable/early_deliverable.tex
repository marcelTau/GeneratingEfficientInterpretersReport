\documentclass{article}
\title{Early Deliverable of \texttt{Generating Efficient Interpreters}}
\author{
    Marcel Taubert  (mt652)\\
    20962335        \\
    \\
    School of Computing \\
    MSc Advanced Computer Science\\
    University of Kent \\
    \\
    Supervisor: Dr. Michael Vollmer
}
\usepackage{natbib,hyperref}
\date{\today}
\begin{document}
\maketitle
\clearpage

\section{Introduction}
Generating efficient interpreters is a project aiming to help programming
language developers to be able to quickly build interpreters for their language
in a few simple steps.

At the moment the program consists of a compiler that takes the source code of
the programming language described in the book ~\cite{Pierce:SF1} Imp and
outputs bytecode. The structure of the bytecode and the syntax of the language
will be described in a later section of this paper.

If you have ever written multiple compilers or interpreters yourself you might
have realized that many parts of the program are very repetitive and not too
depended on the syntax of the language that you're building the interpreter
for. Many parts like the code for executing an instruction, fetching an
instruction or the code generation are very similar every time and can be
generated quite easily ~\cite{vmgen}. In addtion to that the user can provide a
configuration file that describes possible missing parts in detail and modify
the generation without writing any code.

Another part that can be automated is optimizations. We can provide multiple
optimizations and let the user define their own in the configuration file. The
user then can choose which of the optimizations they want to include in their
generation. It is easy to see what optimizations work the best or worst by just
generating multiple versions and running benchmarks between them. All of this
can be archived without needing to write a lot of code.

\section{Technical description} 
Currently the compiler is written to work with source code of the programming
language "Imp" ~\cite{Pierce:SF1}. This is a simple imperative programming
language which uses just a few keywords and a clear syntax which makes it
perfect to start with in this project. The future work section of this paper
will go into detail on how we can extend this project to work with a real
programming language.

This is an example of Imp.
\begin{verbatim}
    Z := X;
    Y := 1;
    while Z != 0 do
        Y := Y * Z;
        Z := Z - 1;
    end
\end{verbatim}

The output of the compiler is a custom bytecode. It is designed to work with a
stack-based virtual machine.

The compiler itself is implemented in the Rust programming language. The
following is the representation of a single bytecode instruction as a Rust
enum.

\begin{verbatim}
pub enum ByteCode {
    Push(usize),
    Pop,
    Add,
    Sub,
    Mul,
    Var(String),
    Eq,
    NEq,
    Lt,
    Gt,
    Lte,
    Gte,
    And,
    Or,
}
\end{verbatim}

At the moment, the complier is held pretty simple and not all the features that
are possible in "Imp" ~\cite{Pierce:SF1} are implemented. 

Currently none of the control flow featrues are implemented since we're missing
instructions for conditional jumps.

For debugging convenience, we've also added a new keyword ('print') to the
language which prints the content of the given variable.

Since "Imp" does not have scope management, we store all variables in a global
table held in the bytecode generator structure in the rust code.

\section{Future Work}
Since this project is still in the early development phase there is a lot still
missing until we have a working version that can be benchmarked.

The next step is to implement conditional jumps and by that being able
to compile programs that use if-statements or loops.

Then we have to think about different optimizations that we want to implement
and build.

At this point we have a fully working compiler for the "Imp" programming language
and we can start benchmarking it's performance.

One step to get closer to work with a real programming language is to
implement scope management.

Then for example, we can think about modifying the compiler to work with Lua.

\clearpage
\bibliographystyle{plain}
\bibliography{early_deliverable}
\end{document}
